\section{Proposed TNU Tolerant Latch} \label{sec:TNU}

In this section we discuss the implementation of a non-robust triple node upset (TNU) tolerant latch named TNU-latch. The development of a robust TNU latch has been investigated but was fruitless due to the number of possible cases to verify. For example, a latch with two more nodes compared to the HRDNUT giving nine total nodes require the verification of 84 unique cases. Instead, we focused on the development of a simple and efficient design that has [TRANSISTOR NUM]. This design has been verified for all possible TNU cases and is shown to be fully tolerant. To our knowledge, this design is the first fully TNU tolerant latch proposed in literature. While it is not clear if current process generation benefit more from interleaving SEU and DNU tolerant designs or directly using a TNU latch, it can be inferred that smaller processes in highly radioactive environments will benefit from a TNU tolerant latch.

First we will describe the design of TNU-latch. The latch consists of a base block latch as in the HRDNUT but with 5 storage blocks. Each storage block has a C-element with 4 inputs with each input connected to the other nodes. In addition 4 of the C-elements have two transistors connected to \textit{CLK} and \textit{CLKB} to ensure that the output node is set to high impedance during the transparent mode. Similar to the HRDNUT, this reduces the power consumption and the delay for a relatively small increase in area. To vote on the output, the nodes in the block latch are connected to two 3-input C-elements. These C-elements drive and 2-input output C-element. A schematic of the TNU-latch is given in [REFERENCE FIG].



The basis behind the latch is to first ensure that the C-elements in the block latch cannot be driven to an incorrect value due to a TNU. To ensure this, each C-element has four inputs. If the latch was designed as in the block based latch for the HRDNUT, the output C-element would have 5 inputs. However, among experimentation with this design it was found that an error on the output element would lead to an unrecoverable error. To solve this issue, the output C-element was split into 2 3-input elements which drive a 2-input element. The is effectively removed the error since a TNU can, in the worst case, only flip a single C-element leaving one output element unaffected thus holding the data. This also allows for the latch to tolerate a TNU with an error on the output since only two errors will affect the internal nodes. None of the C-elements can be flipped due to an internal DNU thus allowing for the output to recover.

First we will evaluate the TNU-latch during the transparent mode. In this mode, the data is loaded to nodes \textit{n1}, \textit{n2}, \textit{n3} and \textit{n4}. This is done when the clock is at a high value which sets the output node to high impedance and turns on the loading pass gates. Once the four nodes are loaded, all the inputs on C-element \textit{C5} are set such that \textit{n5} is set to the loaded value. Since nodes \textit{n1-n5} are all loaded, C-elements \textit{C6} and \textit{C7} are set to the loaded value thus driving the output C-element. A waveform of the latch in transparent mode is given in Fig. [REFERENCE FIG].

We will now evaluate the latch for tolerance against soft errors. In the case of a SEU, the latch is tolerant since an SEU cannot change the state of a C-element. Additionally, the latch is DNU tolerant for similar reasons. When a TNU occurs, there are 56 total strike cases. Due to the simple design of the latch, we condense all cases into 6 distinct cases which are given in the following list. Waveforms were generated using equation \ref{qeq} and the same simulation parameters as in Section \ref{Proposed} to model the pulse shape for each individual case. The waveforms are given in Figs. [REFERENCE FIGURES].

\begin{enumerate}
	\item Consider 3 strikes on the internal nodes \textit{n1}, \textit{n2} and \textit{n3}. The errors will all propagate to C-element \textit{C5} but will not cause a change on node \textit{n5} since \textit{n4} is not affected. Additionally, the errors will propagate to the inputs of \textit{C1}, \textit{C2}, \textit{C3} and \textit{C4}. However, since at least \textit{n5} will be unaffected, the C-elements will hold their state. Next, we look at C-elements \textit{C6} and \textit{C7} and note that since \textit{n5} does not change values, the C-elements hold the correct value on nodes \textit{n6} and \textit{n7}. Since these nodes have the correct value, node \textit{OUT} does not change. This analysis holds true for any TNU combination within the set of [\textit{n1}, \textit{n2}, \textit{n3}, \textit{n4} and \textit{n5}].
	
	\item Assume that nodes \textit{n1}, \textit{n2} and \textit{OUT} are struck by a TNU. The errors on the block latch can be treated as a DNU. Since no additional nodes in the block latch are flipped by the error, the C-elements \textit{C6} and \textit{C7} can only be affected by at most two errors. This implies that neither element will possibly flip due to an error. Since nodes \textit{n6} and \textit{n7} are error-free, \textit{C8} will drive \textit{OUT} back to the correct state. All cases where two errors affect the set [\textit{n1, n2, n3, n4, n5}] will resolve similarly.
	
	\item Consider errors on nodes \textit{n1}, \textit{n6} and \textit{n7}. The error on \textit{n1} will propagate to internal elements \textit{n2, n3, n4} and \textit{n5}. However, since only a single error is at the inputs of the C-elements they hold their correct value. Additionally, \textit{C1} is driven back to the error-free value. This allows for nodes \textit{n6} and \textit{n7} to also be recovered. All cases where there is one error on the internal block behave the same as this example. 
	
	\item In the case of a TNU striking nodes \textit{n1, n2} and \textit{n6}, the errors on \textit{n1} and \textit{n2} propagate to C-elements \textit{C1, C2, C3, C4} and \textit{C5}. Since each C-element is driven by 4 nodes, no additional nodes flip value. This leads to two of the nodes on \textit{C6} having two erroneous inputs. This ensures that the error on \textit{n6} is not recovered. However, \textit{n7} remains error-free thus allowing the output to be fully recovered. Any TNU combination that has two errors from the set [\textit{n1, n2, n3, n4, n5}] and a single error from \textit{n6} or \textit{n7} will evaluate similarly.
	
	\item Consider errors on nodes \textit{n1}, \textit{n6} and \textit{OUT}. In this case the error on \textit{n1} will be fully recovered as in case 3. Since the error is recovered, all input nodes to \textit{C6} are error-free allowing for full recovery of the node. \textit{n6} and \textit{n7} are also error-free setting the output of \textit{C8} to the correct value. This behave similarly for any case which has one error in the set of [\textit{n1, n2, n3, n4, n5}] a single error on \textit{n6} or \textit{n7} and an error on \textit{OUT}.
	
	\item Lastly, assume errors on \textit{n6, n7} and \textit{OUT}. The TNU-latch will fully recover since all of the inputs of \textit{C6} and \textit{C7} are error-free driving nodes \textit{n6} and \textit{n7} to the correct value. The nodes then drive the output of \textit{C8} to the previous error-free value.
	
\end{enumerate}

