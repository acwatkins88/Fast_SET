\section{Conclusion} \label{sec:conc}
In this paper, novel DNU and TNU tolerant latches were proposed. The proposed HRDNUT latch provided DNU robustness which allows for the design to be used in clock gating schemes. Existing designs that were used in clock gating typically relied on the addition of a weak keeper on the output of the latch. As shown in this paper, this circuitry greatly increases the power, area and delay. The only exception to the weak keeper was the DONUT latch which provides DNU robustness. It was shown in Section \ref{sec:res} that the HRDNUT is more efficient compared to the DONUT by providing 11.3\% lower power consumption, 8.33\% lower delay and 72.5\% lower propagation delay. 

Furthermore, the TNU-Latch was introduced which is the first fully TNU tolerant design. Compared to the HRDNUT, the TNU-Latch has a 2X area overhead and consumes 40\% more power. Based off these findings, it is recommended to use the TNU-Latch in extreme radiation environments for modern process technologies. However, as the transistor feature size continues to shrink, the likelihood of a TNU may become a significant concern for terrestrial designs. In addition to terrestrial applications, there has been a concerted effort to send electronics using modern transistor sizes to space due to the fact that most space-grade hardware is on the order of 10-20 years out of date. Highly robust designs, such as the TNU-Latch, may help alleviate this issue by allowing small transistor processes to be used in current processor designs.