\IEEEraisesectionheading{\section{Introduction}\label{sec:introduction}}
% Computer Society journal (but not conference!) papers do something unusual
% with the very first section heading (almost always called "Introduction").
% They place it ABOVE the main text! IEEEtran.cls does not automatically do
% this for you, but you can achieve this effect with the provided
% \IEEEraisesectionheading{} command. Note the need to keep any \label that
% is to refer to the section immediately after \section in the above as
% \IEEEraisesectionheading puts \section within a raised box.

% 
% Here we have the typical use of a "T" for an initial drop letter
% and "HIS" in caps to complete the first word.
\IEEEPARstart{A}{s} the transistor feature size continuously scales down to improve performance, modern circuitry continues to become more susceptible to radiation induced errors commonly referred to as a soft error. Terrestrial soft errors can manifest from either neutron particles originating from cosmic rays or alpha particles from packaging. In space, a soft error can come from gamma rays, protons, neutrons, electrons and heavy ions \cite{Zick2008}. A soft error of any source occurs when a particle hits the diffusion region of a reverse bias transistor. This, in turn, allows an "off" transistor to temporarily conduct current which can cause a voltage change in a node connected to the affected transistor. If the error occurs in combinational logic, the resulting voltage pulse may be propagated to a circuit output and capture by a flip-flop thus causing an error. Additionally, the error my occur directly on an internal latch of a flip-flop causing immediate data corruption. Due to this possibility, there is a need for new latch designs that can tolerate errors.

There has been extensive research in the field of hardening latches against single even upsets (SEU). The most straight forward hardening design is the use of triple modular redundancy (TMR). This design consists of 3 standard latches connected to a 3-input majority voting circuit. While this design is robust against errors, it has the drawback of high area, delay and power consumption. For this reason there have been many other designs proposed that offer high SEU reliability with lower area, delay and power consumption. The first and most common design is the DICE cell proposed in \cite{DICE}. The design in \cite{DICE} consists of 4 one-input c-elements connected in series. Two of the nodes are connected to a pass-gate which allows data to be loaded. When an error occurs in the DICE latch, the struck node is flipped. The two nodes connected to the struck node are set to a high impedance state and one node remains error-free. Since the error-free node is strongly driven to the correct value, it drives the remaining nodes back to the correct value. While this design work well for a single node, it is not capable of handling multi-node errors. 

While the DICE latch is efficient in area, it suffers from high delay. For this reason there has been multitude of SEU tolerant devices that have been proposed. The SEU tolerant designs follow one of two approaches to hardening: sizing transistors such that the critical charge exceeds the maximum injected charge for the intended environment and by designing circuits that functionally tolerate the error. For the former designs, such as \cite{NicoFeedback}, they are typically performance and area efficient. The drawback with these types of designs is that they require accurate estimates for the maximum injected charge. If the maximum charge is found to be too high, a designer using this type of latch would have to choose between performance and reliability. 

The latter type of latch, such as \cite{HIPER, FERST, Hazucha, SEMULatch, Multivdd, BISER}, have the advantage of recovering from a SEU regardless of the injected charge due to the logical functions of the latch forcing recovery of affected node. In cases where the maximum injected charge is not excessively high, the latches have higher performance and area overheads. However, these type of latches are preferable in many cases since the maximum charge may be unknown or very high. 

In modern processes, the transistor size is small enough that a radiation particle may strike multiple simultaneous transistors. Cases where this type of strike may occur are commonly referred to as a single event multiple upset (SEMU). In addition to the SEMU case, high radiation environments may allow for the manifestation of a multiple event multiple upset (MEMU). In this case, multiple radiation particles strike internal transistors simultaneously. When either a SEMU or MEMU strike a latch, they may upset multiple nodes. If two nodes are upset in the latch, this is referred to as a double node upset (DNU). If three nodes are upset, this is called a triple node upset (TNU). The DNU is currently of great concern as the feature size has allowed for a sharp increase in the occurrence of DNUs. Section (REFERENCE SECTION HERE) provides an overview of all existing DNU latches.  To the author's knowledge, there has been no research done on the occurrence of the TNU. It can be inferred however, that future processes or high radiation environments, such as a mission to Mars or Europa, will allow for the TNU to be of large concern.     

To save power, many modern circuit designs employ a technique commonly referred to as clock gating to further reduce the power consumption. Clock gating consists of shutting off the clock to a stable value or "gating" the clock. If clock gating is used in a latch, it may need to hold the stored value for many clock cycles. If the latch is struck by a radiation error while gated, it could lead to a loss of data. This may occur if the latch has high-impedance states after an error. If an error occurs, the high-impedance nodes may slowly discharge causing a loss in data. To remedy this issue, researchers have proposed the addition of output circuitry to hold the data. However, as shown in Section (REFERENCE SECTION HERE), the additional circuitry adds a large overhead to the delay and power consumption.

To solve this problem, we propose the HRDNUT (Highly Robust Double Node Upset Tolerant) latch which is an efficient DNU tolerant design that is capable of recovering all nodes after an error occurs. The recovery feature provides a distinct advantage over previous designs in cases where clock gating is used since it removes the need for the additional circuit since no nodes are held to a high-impedance state after an error. Designs that are DNU tolerant and exhibit this behavior are referred to as DNU-robust. Any design that is DNU tolerant and does has high-impedance states is referred to a DNU-non-robust. The proposed design is throughly compared to existing designs and is more efficient than the existing DNU-robust design in \cite{DONUT} in power, delay and area. The design is also compared to all existing DNU tolerant latches and the most common SEU tolerant latches.

In addition to the DNU latch, we also propose the  TNU-latch which is a TNU tolerant latch that is based on the HRDNUT. While this latch is non-robust, it provides a simple and efficient solution suitable for high reliability applications. To the author's knowledge, the TNU latch is the first of its kind. 

Lastly, even though there are masking factors in combination logic, depending on the design, the error-rate in combinational logic is not insignificant. In cases where an error occurs, the resulting voltage pulse may propagate to the circuit output. As proposed in \cite{FERST}, one approach to remove this type of error is to apply a filtering circuit on the combinational logic output. In the case of mult-node upsets, the design in \cite{FERST} is vulnerable to error. An improved design is also proposed. 

The paper is organized as follows: Section [REFERENCE] provides a discussion on existing DNU tolerant latches, Section [REFERENCE] discusses the HRDNUT, Section [REFERENCE] gives the TNU-latch, Section [REFERENCE] consists of the explanation for the pulse filtering circuit, Section [REFERENCE] contains a comparison of the proposed latches to many existing designs and Section [REFERENCE] concludes the paper.