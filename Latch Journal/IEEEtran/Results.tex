\section{Simulation Results} \label{sec:res}

The proposed latch designs were implemented using the 1.05V 32nm PTM library \cite{PTM} and simulated in HSPICE. All transistor widths for all designs were set to minimum size which is 80nm for PMOS and 40nm for NMOS. All designs were operated at 1 Ghz. We compared the HRDNUT and TNU-latch to existing SEU and DNU tolerant designs. We did not compared to TNU tolerant designs since no other designs are known to exist. For the analysis, we compare to the following SEU tolerant latches: DICE \cite{DICE}, FERST \cite{FERST} and HIPER \cite{HIPER}. Additionally, we also compare to the following DNU tolerant designs: DNCS \cite{DNCS}, Interception \cite{Inter}, HSMUF \cite{HSMUF} and DONUT \cite{DONUT}. All transistors for the implemented latches were set to minimum width and length except for the designs that use a C-element with a weak keeper. In these designs the C-element's PMOS width was set to W=320nm and the NMOS width was set to W=160nm and the weak keeper was sized to be at minimum width. The C-element was sized so that the output driving strength did not allow the keeper to drive an erroneous value in the event of an error. 

For the comparison we measure the propagation delay, average power consumption and area of all design. We then categorize the designs based on the number of errors they can tolerate and if they are robust to the number of errors they tolerate. The delay for each design was calculated based on the difference between the time that input \textit{D} is at $0.5*V_{DD}$ and the output at the same point. The average power was calculated over a 200 ns duration when the latch is error-free. For the calculation of the area, the unit size transistor (UST) metric as adopted in \cite{DNCS} was used. This metric quantifies the expected area based the total transistor area divided by the unit size. For this case, the unit size was set to be 40 nm. Table \ref{table:rtable} gives the results of the simulation.

\begin{table}[h]
\begin{center}
	\caption{SPICE Simulations of Existing Latches using the 1.05V 32nm PTM library }
	\label{table:rtable}
	\begin{tabular}{|m{5em}|m{3.5em}|m{3em}|m{3em}|m{3em}|m{3em}|}
	\hline
	Latch & DNU Immune & DNU Robust & Power ($\mu$W) & Delay (ps) & Area (UST)\\ 
	\hline
	DICE & No & No & 1.332 & 8.145 & 16 \\
	\hline
	FERST & No & No & 3.178 & 31.648 & 60 \\
	\hline
	HIPER & No & No & 1.292 & 2.221 & 27 \\
	\hhline{|=|=|=|=|=|=|}
	DNCS & Yes & No & 4.948 & 22.486 & 61 \\
	\hline
	\cite{Inter} & Yes & No & 5.606 & 79.168 & 89 \\
	\hline
	HSMUF & Yes & No & 1.871 & 1.0626 & 51 \\
	\hline
	HSMUF (Keeper) & Yes & No & 3.787 & 3.945 & 78 \\
	\hhline{|=|=|=|=|=|=|}
	DONUT \cite{DONUT} & Yes & Yes & 4.021 & 14.722 & 54 \\ 
	\hline
	DONUT-M (Section \ref{sec:DNUdes}) & Yes & Yes & 2.760 & 8.421 & 72\\
	\hline
	HRDNUT (Proposed) & Yes & Yes & 2.450 & 2.310 & 66 \\
	\hline
	TNU-Latch & Yes & No & 3.8994 & 46.89 & 123 \\
	\hline
	\end{tabular}
\end{center}
\end{table}

According to Table \ref{table:rtable} the only DNU robust designs are the two DONUT latch implementations and the HRDNUT. Compared to the modified DONUT latch, the HRDNUT provides DNU robustness while reducing the power consumption and number of transistors by 11.3\% and 8.33\% respectively while also reducing the delay by 72.5\%. For the above reasons, the HRDNUT is the best design for clock gating applications due to its high robustness, even after a DNU occurs, and lower power, delay and area overheads.