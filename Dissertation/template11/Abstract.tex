%---------------------------------------------------------------
% The "*" following chapter or section commands omits chapter/
% section numbers.  It also does not include the chapter/section
% in the table of contents -- the \addcontentsline can be used
% to manually force its entry.
%---------------------------------------------------------------
%%%%%%
%%%%%%  This page needed only if Thesis or Dissertation
%%%%%%  Change title of this page to reflect type of paper
%%%%%%


%% Change title below to reflect Thesis or Dissertation

\chapter*{AN ABSTRACT OF THE DISSERTATION OF}
\addcontentsline{toc}{chapter}{Abstract}

% DATE below is the date of your defense
% Change those words in all caps below
Adam Watkins, for the Doctor of Philosophy degree in Electrical and Computer Engineering,
presented on November 2nd, at Southern Illinois University Carbondale.
(Do not use abbreviations.)

\vspace{14pt}
\noindent
TITLE: ANALYSIS AND MITIGATION OF MULTIPLE RADIATION INDUCED ERRORS IN MODERN CIRCUITS
% Change title above and put all in CAPS

\vspace{14pt}

\noindent
MAJOR PROFESSOR: Dr.\ S.\ Tragoudas
%Change professor above

\vspace{14pt}

Due to technology scaling, the probability of a high energy radiation particle striking multiple transistors has continued to increase. This, in turn has created a need for new circuit designs that are can tolerate multiple simultaneous errors. A common type of error in memory elements is the double node upset (DNU) which has continued to become more common. All existing DNU tolerant designs either suffer from high area and performance overhead, may lose the data stored in the element or are vulnerable to an error after a DNU occurs which makes the devices unsuitable for clock gating. In this dissertation, a novel latch design is proposed in which all nodes are capable for fully recovering their correct value after a single or double node upset which is referred to as DNU robust. The proposed latch offers lower delay, power consumption and area requirements compared to existing DNU robust designs.


\newpage

