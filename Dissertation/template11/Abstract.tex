%---------------------------------------------------------------
% The "*" following chapter or section commands omits chapter/
% section numbers.  It also does not include the chapter/section
% in the table of contents -- the \addcontentsline can be used
% to manually force its entry.
%---------------------------------------------------------------
%%%%%%
%%%%%%  This page needed only if Thesis or Dissertation
%%%%%%  Change title of this page to reflect type of paper
%%%%%%


%% Change title below to reflect Thesis or Dissertation

\chapter*{AN ABSTRACT OF THE DISSERTATION OF}
\addcontentsline{toc}{chapter}{Abstract}

% DATE below is the date of your defense
% Change those words in all caps below
Adam Watkins, for the Doctor of Philosophy degree in Electrical and Computer Engineering,
presented on November 2nd, at Southern Illinois University Carbondale.
(Do not use abbreviations.)

\vspace{14pt}
\noindent
TITLE: ANALYSIS AND MITIGATION OF MULTIPLE RADIATION INDUCED ERRORS IN MODERN CIRCUITS
% Change title above and put all in CAPS

\vspace{14pt}

\noindent
MAJOR PROFESSOR: Dr.\ S.\ Tragoudas
%Change professor above

\vspace{14pt}

Due to technology scaling, the probability of a high energy radiation particle striking multiple transistors has continued to increase. This, in turn has created a need for new circuit designs that are can tolerate multiple simultaneous errors. A common type of error in memory elements is the double node upset (DNU) which has continued to become more common. All existing DNU tolerant designs either suffer from high area and performance overhead, may lose the data stored in the element or are vulnerable to an error after a DNU occurs which makes the devices unsuitable for clock gating. In this dissertation, a novel latch design is proposed in which all nodes are capable for fully recovering their correct value after a single or double node upset which is referred to as DNU robust. The proposed latch offers lower delay, power consumption and area requirements compared to existing DNU robust designs.

Multiple simultaneous radiation induced errors are a current problem that must be studied in combinational logic. Typically, simulators are used early in the design phase which use a netlist and rudimentary information of the process parameters to determine the error rate of a circuit. Existing simulators are able to accurately determine the effects when the problem space is limited to one simultaneous error. However, existing methods do not provide accurate information when multiple concurrent errors occur due to inaccurate approximation of the glitch shape when multiple errors meet at a gate. To improve existing error simulation, a novel analytical methodology to accurate determine the pulse shape when two simultaneous errors occur is proposed. Through extensive simulations, it was shown that the proposed methodology matches closely with HSPICE while providing a speedup of 15X.

The analysis of the soft error rate of a circuit has continued to be a difficult problem due to the calculation of the logical effect on a pulse generated by a radiation particle. The most common existing methods to determine logical effects use either exhaustive input pattern simulation or binary decision diagrams. The problem with both approaches is that simulation of the circuit can intractably time consuming. To solve this issue, a simulation tool is proposed which employs an adaptive partitioning algorithm to reduce the simulation time and space overheads of binary decision diagram based simulation. Compared to existing simulation tools, the proposed tool can simulate larger circuits faster.
\newpage

