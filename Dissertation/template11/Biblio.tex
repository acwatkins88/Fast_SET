
%---------------------------------------------------------------
% \begin{thebibliography} does not automatically generate a
% table of contents entry, so it needs to be manually added.
%
% The "{99}" indicates the maximum number of bibliography entries
% and is used to reserve space when indenting the reference
% number for each.  E.g., {9} reserves one number space for
% reference numbers and should be used for less then 10 entries,
% {99} reserves two number spaces, {999} reserves three.
%
% "\frenchspacing" can be used so that periods following
% author initials are not given extra space (as an end of
% sentence period would).
%
% Each reference is started with \bibitem{xxxxxx} where {xxxxxx}
% is a unique label assigned to that reference.  You can then
% refer to that reference in the text using the \cite command.
% E.g. "... as seen in Rasulov \cite{RasulovB}..."
% might be formatted as "... as seen in Rasulov [7]..."
% Using reference labels and the \cite command eliminates the
% need to manually keep track of reference numbers.
% Note:  these labels are case sensitive ("rasulovb" will
%        not match "RasulovB") and must be unique.
%
% Note the use of the "\bysame" definition used in the last
% reference -- this should be used in place of the author's
% name when a reference follows another by the same author.
%---------------------------------------------------------------


\begin{thebibliography}{99}
\addcontentsline{toc}{chapter}{References}

\frenchspacing


\bibitem{Anton}
Anton, H.,
\emph{Elementary Linear Algebra},
 John Wiley \& Sons, New York, 1977.

\bibitem{Huang}
Huang, X. and Krantz, S.G.,
\emph{On a problem of Moser},
Duke Math. J.
{\bf 78} (1995), 213--228.

\bibitem{Kellog}
Kellog, O.D.,
\emph{Harmonic functions and Green's integral},
Trans. Amer. Math. Soc.
{\bf 13} (1912), 109--132.

\bibitem{Kruzhilin}
Kruzhilin, N.G.,
\emph{Local automorphisms and mappings of smooth strictly
pseudoconvex hypersurfaces},
(Russian),
Dokl. Akad. Nauk SSSR
{\bf 271} (1983), 280--282.

\bibitem{Macdonald}
Macdonald, I.G.,
\emph{Symmetric Functions and Hall Polynomials},
Second edition, Clarendon, Oxford, 1995.

\bibitem{RasulovA}
Rasulov, K.M.,
Dokl. Akad. Nauk SSSR {\bf 252} (1980), 1059--1063; English transl. in Soviet Math. Dokl.
{\bf 21} (1980).

\bibitem{RasulovB}
Rasulov, K.M.,
%\bysame,
Dokl. Akad. Nauk SSSR {\bf 270} (1983), 1061--1065; English transl. in Soviet Math. Dokl.
{\bf 27} (1983).


\end{thebibliography}
