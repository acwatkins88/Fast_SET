
%---------------------------------------------------------------
% \begin{thebibliography} does not automatically generate a
% table of contents entry, so it needs to be manually added.
%
% The "{99}" indicates the maximum number of bibliography entries
% and is used to reserve space when indenting the reference
% number for each.  E.g., {9} reserves one number space for
% reference numbers and should be used for less then 10 entries,
% {99} reserves two number spaces, {999} reserves three.
%
% "\frenchspacing" can be used so that periods following
% author initials are not given extra space (as an end of
% sentence period would).
%
% Each reference is started with \bibitem{xxxxxx} where {xxxxxx}
% is a unique label assigned to that reference.  You can then
% refer to that reference in the text using the \cite command.
% E.g. "... as seen in Rasulov \cite{RasulovB}..."
% might be formatted as "... as seen in Rasulov [7]..."
% Using reference labels and the \cite command eliminates the
% need to manually keep track of reference numbers.
% Note:  these labels are case sensitive ("rasulovb" will
%        not match "RasulovB") and must be unique.
%
% Note the use of the "\bysame" definition used in the last
% reference -- this should be used in place of the author's
% name when a reference follows another by the same author.
%---------------------------------------------------------------

\bibliographystyle{plain}
\bibliography{Bibfile.bib}